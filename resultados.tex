%TCIDATA{LaTeXparent=0,0,relatorio.tex}
\chapter{Estudo de Caso}\label{resultados}

Havendo constitu�do uma base de registros significativa, com exatas 346 publica��es analisadas � partir de crit�rios de inclus�o e exclus�o, o que resultou num total de 210 aceitas e classificadas. Tais n�meros se referem �queles atendidos por somente um grupo entre os participantes, sendo importante ressaltar o algoritmo adotado na divis�o feita, que distribuiu sequencialmente e por ordem alfab�tica do t�tulo uma publica��o para cada grupo at� que todas as 1034 estivessem associadas a um dos tr�s grupos cadastrados.

Assim, os seguintes resultados representam uma parte, mais precisamente um ter�o da amostra total. O tipo de distribui��o feita permite com que dentro de cada subamostra pertencente aos grupos haja publica��es em praticamente todos os t�picos, com uma desprez�vel varia��o dada a rela��o entre t�pico abordado e t�tulo e a grande quantidade de publica��es contempladas.

A inten��o de todos os envolvidos no trabalho � de que haja a continuidade de an�lises at� que o total da amostra principal aceito pelos crit�rios de inclus�o tenha sido classificado, o que dar� mais import�ncia e robustez aos resultados obtidos pelo mapeamento de estudos.

\section{Apresenta��o dos Gr�ficos}

