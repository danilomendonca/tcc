\chapter{Simula��es}\label{desempenho}

Foi simulado um sistema de comunica��o �ptico que utiliza modula��o DQPSK com multiplexa��o por polariza��o e detec��o coerente. Os s�mbolos gerados s�o corrompidos por ru�do do tipo ASE e ru�do de fase modelado pelo processo discreto de Wiener. Os esquemas de recupe��o de portadora foram implementados em MATLAB e o desempenho do sistema foi medido utilizando processos de Monte Carlo. 

A Figura \ref{fig:penalidade} mostra o impacto do ru�do de fase $\Delta\nu T_s$ em cada algoritmo para uma taxa de erro de bit (BER) de $10^{-3}$.

\begin{figure}[h]
	\centering
	\includegraphics[width=0.80\textwidth]{figs/penalidade.eps}
	\caption{Curvas de penalidade em dB para taxa de erro BER de $10^{-3}$.}
	\label{fig:penalidade}
\end{figure}

O desempenho do esquema DD mostra-se o pior. Ele � mais sens�vel ao ru�do de fase, requerendo uma maior SNR para a mesma taxa de erro que seu concorrente. Em um enlace POLMUX DQPSK com taxa de 28 Gbaud (112 Gb/s) e lasers com largura de linha de 1 MHz, $\Delta\nu T_s=4.45\times10^{-4}$ o DD oferece uma penalidade de aproximadamente 0,7 dB por exemplo.
