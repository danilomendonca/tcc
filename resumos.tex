%TCIDATA{LaTeXparent=0,0,relatorio.tex}
\resumo{Resumo}{

A Computa��o Orientada a Servi�os surgiu com o intuito de prover maior efici�ncia � produ��o, provis�o e consumo de recursos computacionais, especialmente os \textit{softwares}, que passam a compor unidades coesas e granulares de l�gica capazes de se intercomunicarem e de formarem novas solu��es por meio de sua composi��o em novos rearranjos, aumentando o reuso, a agilidade, o retorno de investimento e o alinhamento da TI com os processos de neg�cio. Dada a incerteza a respeito do estado das pesquisas relacionadas ao tema, detectou-se a necessidade de uma classifica��o de estudos que possibilite identificar de forma esquem�tica os t�picos existentes e que aponte tend�ncias, atores, tipos de pesquisa e quais sub�reas receberam maior �nfase em detrimento daquelas que ainda carecem de avan�os. A partir do Systematic Mapping Study, que envolve a busca em sistemas de registro de publica��es em f�runs de interesse de modo a classifica-las segundo categorias escolhidas, avaliamos publica��es que tratam da qualidade de servi�os na Computa��o Orientada a Servi�os. Como resultado, os gr�ficos obtidos no formato \textit{bubble plot} oferecem informa��es �teis e importantes para a compreens�o do estado da pesquisa da �rea focada nos �ltimos 10 anos, entre elas a sua imaturidade, comprovada pela maior frequ�ncia de pesquisas do tipo solu��o e na falta de pesquisas que as validem.
Por �ltimo e n�o menos importante, avaliou-se o desempenho e utilidade da ferramenta de suporte desenvolvida para atender a este mapeamento de estudos e que demonstrou superar expectativas, apontando para seu potencial uso em outros trabalhos de mapeamento e revis�o de literatura.

}

\resumo{Abstract}{\textit{English version}}
