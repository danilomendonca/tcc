%TCIDATA{LaTeXparent=0,0,relatorio.tex}
\resumo{Resumo}{

A Computa��o Orientada a Servi�os surgiu com o intuito de prover maior efici�ncia � produ��o, provis�o e consumo de recursos computacionais, especialmente os \textit{softwares}, que passam a compor unidades coesas e granulares de l�gica capazes de se intercomunicarem e de formarem novas solu��es por meio de sua composi��o em novos rearranjos, aumentando o reuso, a agilidade, o retorno de investimento e o alinhamento da TI com os processos de neg�cio. Dada a incerteza a respeito do estado das pesquisas relacionadas ao tema, detectou-se a necessidade de uma classifica��o de estudos que possibilite identificar de forma esquem�tica os t�picos existentes e que aponte tend�ncias, atores, tipos de pesquisa e quais sub�reas receberam maior �nfase em detrimento daquelas que ainda carecem de avan�os. A partir do Systematic Mapping Study, que envolve a busca em sistemas de registro de publica��es em f�runs de interesse de modo a classifica-las segundo categorias escolhidas, avaliamos publica��es que tratam da qualidade de servi�os na Computa��o Orientada a Servi�os. Como resultado, os gr�ficos obtidos no formato \textit{bubble plot} oferecem informa��es �teis e importantes para a compreens�o do estado da pesquisa da �rea focada nos �ltimos 10 anos, entre elas a sua imaturidade, comprovada pela maior frequ�ncia de pesquisas do tipo solu��o e na falta de pesquisas que as validem.
Por �ltimo e n�o menos importante, avaliou-se o desempenho e utilidade da ferramenta de suporte desenvolvida para atender a este mapeamento de estudos e que demonstrou superar expectativas, apontando para seu potencial uso em outros trabalhos de mapeamento e revis�o de literatura.

}

\resumo{Abstract}{\textit{English version}}

The Service Oriented Computing has emerged to address the need of an improved efficiency in the production, supply and consuming of computational resources, especially the software resources, which begin to compose granular and cohesive units of logic capable of intercommunicate and to form new solutions through its composition in new arrangements, increasing the reuse, agility, return of investment and aligning of IT with business. Given the uncertainty about the status of related research, the need for a classification of studies that allows identify schematically the existing topics and that points the trends, actors, types of research and what sub-areas have received greater emphasis over those that still require improvements. From a Systematic Mapping Study, which involves the search in registry systems of forums of interest for publications in order to classify them according to categories chosen, we evaluated the publications dealing with quality of services in Service Oriented Computing. As a result, the graphs obtained in the form of bubble plots provide useful and important information for understanding the state of research in focus in the last 10 years, among them its low maturity level, evidenced by increased frequency of solution type research and the lack of research to validate them. Last but not least, we evaluated the performance and usefulness of the support tool developed to attend this mapping study and demonstrated to exceed expectations, pointing to its potential use in other literature mapping and review works.