%TCIDATA{LaTeXparent=0,0,relatorio.tex}
\resumo{Resumo}{

A Computa��o Orientada a Servi�os surgiu com o intuito de prover maior efici�ncia � produ��o, provis�o e consumo de recursos computacionais, especialmente os \textit{softwares}, que passam a compor unidades coesas e granulares de l�gica capazes de se intercomunicarem e de formarem novas solu��es por meio de sua composi��o em novos rearranjos, aumentando o reuso, a agilidade, o retorno de investimento e o alinhamento da TI com os processos de neg�cio. Dada a incerteza a respeito do estado das pesquisas relacionadas ao tema, detectou-se a necessidade de uma classifica��o de estudos que possibilite identificar de forma esquem�tica os t�picos existentes e que aponte tend�ncias, atores, tipos de pesquisa e quais sub�reas receberam maior �nfase em detrimento daquelas que ainda carecem de avan�os. A partir do Systematic Mapping Study, que envolve a busca por estudos em sistemas de registro e busca de publica��es e em f�runs de interesse de modo a classifica-las segundo facetas escolhidas, avaliamos artigos que tratam da qualidade de servi�os na Computa��o Orientada a Servi�os. Como resultado...

}

\resumo{Abstract}{\textit{English version}}

The work here presented shows how to build applications that perform data collection and analysis of virtual social networks, focusing on Facebook. They are also summarized ways of carrying out the processes of collecting, analyzing and building applications and presented the concepts involved in understanding these processes.

The work is divided into five chapters: Introduction, Theoretical Background, Implementation, Case Studies and Conclusion. In the Introduction chapter, all chapters of this work are discussed superficially. It also addresses the goals are divided into primary objectives, general, specific and long-term work. This chapter also presents the motivations that are all justified and is shown the importance of proper work. Although this chapter describes related work. Theoretical Background In chapter presents the main concepts of work that are necessary for understanding the ensuing chapters. Implementation is presented in chapter created the application, what it does and how. In addition to showing the reader the aspects involved in software development. In the penultimate chapter, case study presents the analysis performed from the saved data categorization. In the last chapter, Conclusion, reflects on the work and the achievement of objectives, addressing some of the content described in all chapters of the work. Then we have four annexes relating to the Implementation Chapter.
